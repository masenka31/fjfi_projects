\documentclass[a4paper, 12pt, fleqn]{article}
\usepackage[czech]{babel}
\usepackage[utf8x]{inputenc}
\usepackage[T1]{fontenc}
%\usepackage[latin2]{inputenc} % pro iso8859-2
%\usepackage[IL2]{fontenc}     % fonty vygenerované pro iso8859-2

%%%%%%%%%%%%%%%%%%%%%%%%%%%%%%%%%%%%%%%%%%%%%%%%%%%%%%%%%%%%%%%%%%%%%%%%%
%% Hlavička ZDE VYPLNIT 
\newcommand{\autor}{Daniel KARLÍK, Michaela MAŠKOVÁ}
\newcommand{\cislomp}{1} % 1,2,3
\newcommand{\zadani}{OVM model}
\newcommand{\rocnik}{2020/2021}
%% Konec hlavičky
%%%%%%%%%%%%%%%%%%%%%%%%%%%%%%%%%%%%%%%%%%%%%%%%%%%%%%%%%%%%%%%%%%%%%%%%%


%\usepackage{enumitem} 
%\setlist{noitemsep, nolistsep}

\usepackage{calc}
\setlength{\textheight}{10in}
\setlength{\textwidth}{6.5in}
\setlength\oddsidemargin{0cm}
\setlength\evensidemargin{(0cm}
\setlength\topmargin{(\paperheight-\textheight-\headheight-\headsep-\footskip)/2 - 1in}

%% matematika
\usepackage{amssymb}
\usepackage{amsmath}

%% grafika
\usepackage{color}
\usepackage{tikz}
\usetikzlibrary{decorations.pathreplacing, patterns}
\usetikzlibrary{arrows.meta,positioning, datavisualization}

%% Nastavení grafiky matlabovských kódů ------------------------------------------------
\usepackage{listings} % balíček pro blok kódu
\usepackage{matlab-prettifier} % matlabovské barvičky atd.
\renewcommand{\lstlistingname}{Kód} % úprava caption
% nějaké další drobnosti
\lstset{basicstyle=\mlttfamily}
\lstset{frame=single}
\lstset{numbers=left}

% pro případ použití speciálních českých znaků v kódu (např. v komentáři)
\lstset{literate=
	{í}{{\'i}}1
	{á}{{\'a}}1
	{ý}{{\'y}}1
	{é}{{\'e}}1
	{ř}{{\v{r}}}1
	{ó}{{\'o}}1
	{ů}{{\r{u}}}1
	{č}{{\v{c}}}1
	{ě}{{\v{e}}}1
}

% ---------------------------------------------------------------------------------------
\begin{document}
%%%%%%%%%%%%%%%%%%%%%%%%%%%%%%%%%%%%%%%%%%%%%%%%%%%%%%%%%%%%%%%%%%%%%%%%%%%%%%%%
%% Vysázení hlavičky - NEMĚNIT
\hrule
\medskip
\noindent{\Large \textbf{01SSI} -- Miniprojekt číslo \cislomp\hfill\rocnik}
\hrule
\medskip
{\large
\noindent
\begin{tabular}{lp{12cm}}
Posluchači:& \textbf{\autor}\\
Zadání:& \textbf{\zadani} \\
\end{tabular}

\medskip
\noindent
 \hspace{2mm}Odevzdáno: \hfill Získané body: \hfill Finální: ANO/NE
}
\medskip
\hrule
%% Konec vysázení hlavičky
%%%%%%%%%%%%%%%%%%%%%%%%%%%%%%%%%%%%%%%%%%%%%%%%%%%%%%%%%%%%%%%%%%%%%%%%%%%%%%%%%%%%

%% Vlastní popis

\section*{Ukázka matlabovského kódu}
Takhle to pak vypadá, když chci přidat kus kódu:

\begin{lstlisting}[style=Matlab-editor,caption={Ukázka Matlab kódu.},captionpos=b,label={Kód: test}]
% fit
pd_GEV = fitdist(X,'GeneralizedExtremeValue')
pd_LN = fitdist(X, 'LogNormal')

% KS test
[h1,p1] = kstest(X,'CDF',pd_GEV)
h1 = 0
p1 = 0.9969

[h2,p2] = kstest(X,'CDF',pd_LN)
h2 = 0
p2 = 0.8962
\end{lstlisting}

\section*{Popis modelu}
\emph{Nemusí být detailní, stačí stručně. Podrobně rozepsat prvky modelu, které se neřešily na přednášce.}\\

Exclusion process s náhodným updatem
\begin{enumerate}
	\item Náhodně vyberu buňku $x\in\mathbb{L}$. 
	\item Je-li $x$ obsazena částicí, přeskočí tato s pravděpodobností $p(x,y)$ do $y\in\mathbb{L}$.
\end{enumerate}

\noindent Konkrétní tvar $p(x,y)$ je
\[
	p(x,y)=\begin{cases}
		\alpha/\|x-y\| & 0<\|x-y\|\leq R\,,\\
		0 & \|x-y\|> R\,,\\
		1-\sum_{y\neq x} p(x,y) & \|x-y\|=0 \,.
	\end{cases}
\]

\noindent Parametry $\alpha\in[0,\infty]$, $R\in\mathbb{N}$.

\section*{Stacionární řešení}
\emph{Stačí report o zvoleném postupu a metodách. V případě náročných výpočtů stačí dodat psané rukou na papíře.}\\

Hledám ve tvaru
\[
	\pi(\tau)=\frac{f(\tau)}{\sum_\sigma f(\sigma)}\,,\quad f(\tau)=\prod_x f_x^1(\tau(x))\,.
\]
Platí $p(x,y)=p(y,x)$ pro všechna $x,y\in\mathbb{L}$. Rovnici
\[
	p(x,y)f_x^1(1)f_y^1(0)=p(y,x)f_y^1(1)f_x^1(0)
\]
řeší $f_x(1)=\varrho$, $f_{x}(0)=1-\varrho$. Všechny konfigurace jsou tedy stejně pravděpodobné, tzn.
\[
	\pi(\tau)={L-2\choose N-1}{L\choose N}^{-1}
\]

\section*{Manuál}
\emph{Jak spustit program, kde nastavit parametry, ....}\\

GUI se spouští skriptem \verb+RUN.m+:
\begin{center}
\begin{tikzpicture}
	\draw (0,0) rectangle (10,5);
	\node[draw] at (2,1) {Button 1};
	\node[draw] at (2,2.5) {Button 2};
	\node[draw] at (2,4) {Button 3};
	\node[draw] at (7, 1) {|<-------Ruler------->|};
	\draw (5,2) rectangle (9,4.5);
\end{tikzpicture}
\end{center}

\begin{itemize}
	\item Button 1 spouští...
	\item Button 2 spouští...
	\item Button 3 spouští...
	\item Ruler nastavuje...
\end{itemize}

\noindent (Nebo)Parametry se nastavují v hlavičce skriptu \verb+RUN+. Nastavuje se:
\begin{itemize}
	\item $L$ = počet buněk
	\item $N$ = počet částic
	\item ...
\end{itemize}


\section*{Popis programu}
\emph{Ideální je nějaký pseudokód s drobným vysvětlením, jde hlavně o strukturu, nemusíte vysvětlovat všechno!}\\

\noindent Skript \verb+RUN+
\begin{itemize}
	\item Nastavení parametrů $L,N,p$
	\item Počáteční podmínka pomocí funkce \verb+pocpod=POCPOD(L,N,p)+
	\item While-cyklus dokud \verb+flow<flowmax+
	\item Update pomocí funkce \verb+(xnew,vnew)=UPDATE(x,v)+
	\item Vykreslení
\end{itemize}

\noindent Funkce \verb+pocpod=POCPOD(L,N,p)+
\begin{itemize}
	\item Náhodně rozhodí do mřížky $N$ částic.
\end{itemize}

\noindent Funkce \verb+(xnew,vnew)=UPDATE(x,v)+
\begin{itemize}
	\item ...
	\item ...
\end{itemize}

\section*{Porovnání teorie a simulací, výsledky}
\emph{Jednoduše popsat, co se porovnává a porovnat. Vložte obrázky a diskutujte.}\\

Porovnání hustotního profilu simulací ($\circ$) a teorie (red line).

\begin{center}
\begin{tikzpicture}
	\draw (0,0) rectangle (10,5);
	\draw[red] (0,2) -- (10,2);
	\foreach \x in {0,1,...,10} {\draw (\x,2) circle (.1);}
\end{tikzpicture}
\end{center}

\end{document}