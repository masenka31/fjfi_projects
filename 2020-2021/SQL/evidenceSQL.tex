\documentclass[11pt,oneside,american,czech]{book}
\usepackage[T1]{fontenc}
\usepackage[utf8]{inputenc}
\usepackage[a4paper]{geometry}
\geometry{verbose,tmargin=3cm,bmargin=2cm,lmargin=1.5cm,rmargin=1.5cm,headheight=0.8cm,headsep=1cm,footskip=1cm}
% footskip dává mezeru mezi číslování stránek a textem
\pagestyle{headings}
\setcounter{secnumdepth}{3}
\usepackage{url}
\usepackage{amsmath}
\usepackage{amsthm}
\usepackage{amssymb}
\usepackage{graphicx}
\usepackage{setspace}
\usepackage{subfiles}
\usepackage{cite}
\usepackage{tikz}
\usetikzlibrary{shapes,arrows}
\usepackage{babel}

\usepackage[charter]{mathdesign}


% Packages for code snipets
%\usepackage{verbatim}
\usepackage{listings}
\usepackage{matlab-prettifier}
\renewcommand{\lstlistingname}{Kód}

\lstset{basicstyle=\mlttfamily}
\lstset{frame=single}
\lstset{numbers=left}

\lstset{literate=
	{í}{{\'i}}1
	{á}{{\'a}}1
	{ý}{{\'y}}1
	{é}{{\'e}}1
	{ř}{{\v{r}}}1
	{ó}{{\'o}}1
	{ů}{{\r{u}}}1
	{č}{{\v{c}}}1
	{ě}{{\v{e}}}1
}

\title{SQL protokol -- evidence Ultimate Frisbee}
\author{Michaela Mašková}
\date{\today}


\begin{document}

\maketitle

\section*{Představení problému}

Tato databáze má za cíl mapovat českou ultimate frisbee scénu jako evidenci jejích členů společně s nezbytnými informacemi o týmech. Zahrnuje taktéž statistiky jak hráče, tak týmu.

Pár poznámek:

\begin{itemize}
	\item Česká scéna je rozdělená na několik regionů (spádových oblastí). Týmy tak působí v jednotlivých regionech a zároveň mají své domovské město.
	\item Ultimate frisbee se hraje ve třech kategoriích: mix, muži a ženy.
	\item Spirit of the game je zvláštní hodnocení fair play, které má celkem 5 kategorií, v nichž se udává hodnocení mezi 0 a 4. Součet těchto hodnot je celkový výsledek spirita. Spirit se udává vždy pro každý tým za každý odehraný turnaj.
	\item Produktivita hráče se počítá jako počet bodů (skóre + asistence) v průměru na zápas.
\end{itemize}

\section*{Schéma}

\tikzstyle{block} = [rectangle, draw, text width=5em, text centered, rounded corners, minimum height=4em]

\tikzstyle{arrow} = [-latex,thick]
\tikzstyle{arrow2} = [latex-,thick]
\tikzstyle{arrow n1 filled} = [<<-latex, thick]
\tikzstyle{arrow 1n filled} = [latex->>, thick]
\tikzstyle{arrow n1} = [<<->, thick]
\tikzstyle{arrow 1n} = [<->>, thick]
\tikzstyle{arrow 11 right} = [<-latex, thick]
\tikzstyle{arrow 11 left} = [latex->, thick]

\begin{center}
	\begin{tikzpicture}[node distance = 4cm, auto]
	\node[block](hrac) {Hráč};
	\node[block, right of=hrac, xshift=2cm](tym) {Tým};
	\node[block, above of=tym, xshift=4cm, yshift=-2cm](mesto) {Město};
	\node[block, right of=tym, yshift=-1cm](soutez) {Soutěž};
	\node[block, right of=soutez](kategorie) {Kategorie};
	\node[block, right of=mesto](region) {Region};
	\node[block, below of=hrac](statistika hrace) {Statistika hráče};
	\node[block, below of=tym](statistika tymu) {Statistika týmu};
	
	\draw[arrow n1 filled] (hrac) -- node {hraje za} (tym);
	\draw[arrow 1n filled] (mesto) -| node[anchor=south] {působí v} (tym);
	\draw[arrow 1n filled] (tym) -- node[anchor=south]{hraje} (soutez);
	\draw[arrow n1 filled] (soutez) -- node{v rámci} (kategorie);
	\draw[arrow n1 filled] (mesto) -- node{v daném} (region);
	\draw[arrow 11 left] (hrac) -- node{má} (statistika hrace);
	\draw[arrow 11 left] (tym) -- node{má} (statistika tymu);
	
	
	\end{tikzpicture}
\end{center}


\section*{Tabulky}
\subsection*{Přehled}
\tikzstyle{nazev} = [rectangle, draw, text width=8.7em, text centered, minimum height=0.8cm, minimum width=2.8cm]
\tikzstyle{nazev2} = [rectangle, draw, text width=12em, text centered, minimum height=0.8cm, minimum width=2.8cm]
\tikzstyle{table} = [rectangle, draw, text width=8.7em, minimum width=2.8cm, minimum height=3cm]
\tikzstyle{table2} = [rectangle, draw, text width=12em, minimum width=2.8cm, minimum height=3cm]

\begin{tikzpicture}
	\node[nazev](hracnazev) {Hráč};
	\node[table,below of=hracnazev, yshift=-1cm](hraclist) { \vspace{-5mm}
		\begin{itemize}
		\setlength\itemsep{-0.5em}
		\item rodné číslo
		\item jméno
		\item příjmení
		\item číslo dresu
		\item tym.id
		\end{itemize}
	};
	\node[nazev,right of=hracnazev, xshift=2.6cm](tymnazev) {Tým};
	\node[table,below of=tymnazev, yshift=-1cm](tymlist) { \vspace{-5mm}
		\begin{itemize}
		\setlength\itemsep{-0.5em}
		\item id.tym
		\item název týmu
		\item mesto.id
		\end{itemize}
	};
	\node[nazev,right of=tymnazev, xshift=2.6cm](mestonazev) {Město};
	\node[table,below of=mestonazev, yshift=-1cm](mestolist) { \vspace{-5mm}
		\begin{itemize}
		\setlength\itemsep{-0.5em}
		\item id.mesto
		\item název města
		\item region.id
		\end{itemize}
	};
	\node[nazev,right of=mestonazev, xshift=2.6cm](regionnazev) {Region};
	\node[table,below of=regionnazev, yshift=-1cm](regionlist) { \vspace{-5mm}
		\begin{itemize}
		\setlength\itemsep{-0.5em}
		\item id.region
		\item název regionu
		\end{itemize}
	};



	\node[nazev,below of=hraclist, yshift=-1.5cm](souteznazev) {Soutěž};
	\node[table,below of=souteznazev, yshift=-1cm](soutezlist) { \vspace{-5mm}
		\begin{itemize}
		\setlength\itemsep{-0.5em}
		\item tym.id
		\item kategorie.id
		\end{itemize}
	};
	\node[nazev,right of=souteznazev, xshift=2.6cm](kategorienazev) {Kategorie};
	\node[table,below of=kategorienazev, yshift=-1cm](kategorielist) { \vspace{-5mm}
		\begin{itemize}
		\setlength\itemsep{-0.5em}
		\item id.kategorie
		\item název kategorie
		\end{itemize}
	};
	\node[nazev2,right of=kategorienazev, xshift=3.22cm](statistika tymu nazev) {Statistika týmu};
	\node[table2,below of=statistika tymu nazev, yshift=-1cm](statistika tymu list) { \vspace{-5mm}
		\begin{itemize}
		\setlength\itemsep{-0.5em}
		\item tym.id
		\item znalost pravidel
		\item fauly a body contact
		\item férové smýšlení
		\item pozitivní přístup
		\item komunikace
		\end{itemize}
	};
	\node[nazev2,right of=statistika tymu nazev, xshift=3.84cm](statistika hrace nazev) {Statistika hráče};
	\node[table2,below of=statistika hrace nazev, yshift=-1cm](statistika hrace list) { \vspace{-5mm}
		\begin{itemize}
		\setlength\itemsep{-0.5em}
		\item hrac.id
		\item skóre
		\item asistence
		\item počet odehraných zápasů
		\end{itemize}
	};	

\end{tikzpicture}

\subsection*{Vytvoření tabulek}

\begin{lstlisting}[language=SQL]
	
	CREATE TABLE tym(
		tymID INTEGER PRIMARY KEY,
		nazev_tymu TEXT,
		mestoID INTEGER,
		FOREIGN KEY (mestoID) REFERENCES mesto(mestoID)
	);
	
	CREATE TABLE hrac(
		rc INTEGER PRIMARY KEY,
		jmeno TEXT,
		prijmeni TEXT,
		cislo INTEGER,
		tymID INTEGER,
		FOREIGN KEY (tymID) REFERENCES tym(tymID)
	);
	
	CREATE TABLE mesto(
		mestoID INTEGER PRIMARY KEY,
		nazev_mesta TEXT,
		regionID INTEGER,
		FOREIGN KEY (regionID) REFERENCES region(regionID)
	);
	
	
	
	
	CREATE TABLE region(
		regionID int PRIMARY KEY,
		nazev_regionu text
	);
	
	CREATE TABLE kategorie(
		kategorieID INTEGER PRIMARY KEY,
		nazev_kategorie TEXT
	);
	
	CREATE TABLE soutez(
		tymID INTEGER,
		kategorieID INTEGER,
		FOREIGN KEY (tymID) REFERENCES tym(tymID),
		FOREIGN KEY (kategorieID) REFERENCES kategorie(kategorieID)
	);
	
	CREATE TABLE hostovani(
		hracID INTEGER,
		tymID INTEGER,
		FOREIGN KEY (hracID) REFERENCES hrac(hracID),
		FOREIGN KEY (tymID) REFERENCES tym(tymID)
	);
	
	CREATE TABLE statistika_tymu(
		tymID INTEGER,
		pravidla REAL,
		fauly REAL,
		ferovost REAL,
		pozitiv REAL,
		komunikace REAL,
		FOREIGN KEY (tymID) REFERENCES tym(tymID)
	);
	
	CREATE TABLE statistika_hrace(
		hracID INTEGER,
		skore REAL,
		asistence REAL,
		zapasu REAL,
		FOREIGN KEY (hracID) REFERENCES hrac(rc)
	);
	
\end{lstlisting}

\pagebreak
\section*{Views}

\tikzstyle{block dashed} = [rectangle, draw, text width=5em, text centered, rounded corners, minimum height=4em, dashed]
\tikzstyle{no borders} = [text width=8em, text centered]

\subsubsection*{5 nejproduktivnějších hráčů}

Cílem je zobrazit 5 hráčů s nejvyšším skóre produktivity a zároveň ukázat, z jakého pochází týmu. Stejně tak chceme zobrazit i danou statistiku -- tedy počet bodů, asistencí i odehraných zápasů.

\begin{lstlisting}[language=SQL]
	CREATE VIEW nejproduktivnejsiHraci AS
	SELECT
	 jmeno AS Jméno,
	 prijmeni AS Příjmení,
	 (skore + asistence) / zapasu AS Produktivita,
	 nazev_tymu AS Tým,
	 skore AS Skóre,
	 asistence AS Asistence,
	 zapasu AS 'Odehraných zápasů'
	FROM hrac
	INNER JOIN statistika_hrace ON statistika_hrace.hracID = hrac.rc
	INNER JOIN tym ON hrac.tymID = tym.tymID
	ORDER BY Produktivita DESC
	LIMIT 5;
\end{lstlisting}

\subsubsection*{Kolik hráčů hraje v jednotlivých týmech a v jednotlivých městech?}

Cílem je získat views, které vytvoří tabulku týmů společně s aktivním počtem hráčů. Stejná tabulka pak pro jednotlivá města, kde ovšem chceme jak počet hráčů hrajících v jednotlivých městech, tak počet týmů v daném městě.

\begin{lstlisting}[language=SQL]
	CREATE VIEW hracuVTymu AS
	SELECT
	 nazev_tymu AS Tým,
	 count(rc) AS 'Počet hráčů'
	FROM hrac
	INNER JOIN tym ON hrac.tymID = tym.tymID
	GROUP BY nazev_tymu
	ORDER BY count(rc) DESC;
\end{lstlisting}

\begin{lstlisting}[language=SQL]
	CREATE VIEW hracuVeMeste AS
	SELECT
	 nazev_mesta AS Město,
	 count(rc) AS 'Počet hráčů',
	 count(DISTINCT tym.tymID) AS 'Počet týmů'
	FROM tym
	INNER JOIN hrac ON hrac.tymID = tym.tymID
	INNER JOIN mesto ON mesto.mestoID = tym.mestoID
	GROUP BY nazev_mesta
	ORDER BY 'Počet hráčů' DESC;
\end{lstlisting}

\pagebreak
\subsubsection*{Spirit skóre pro týmy}

První view slouží k ukázání všech týmů a jejich průměrného Spirita. Jelikož vyšší Spirit znamená lepší výsledek, chceme je seřadit sestupně.

\begin{lstlisting}[language=SQL]
	CREATE VIEW tymySpiritAll AS
	SELECT
	 nazev_tymu AS Tým,
	 pravidla+fauly+ferovost+pozitiv+komunikace AS Spirit,
	 nazev_mesta AS Město,
	 nazev_regionu AS Region
	FROM tym
	INNER JOIN statistika_tymu ON statistika_tymu.tymID = tym.tymID
	INNER JOIN mesto ON mesto.mestoID = tym.mestoID
	INNER JOIN region ON mesto.regionID = region.regionID
	ORDER BY Spirit DESC;
\end{lstlisting}

Následující view vyprodukuje tabulku týmů jednotlivých regionů s nejvyšším Spirit skóre.

\begin{lstlisting}[language=SQL]
	CREATE VIEW tymyNejvyssiSpiritRegionu AS
	SELECT
	 nazev_tymu AS Tým,
	 pravidla+fauly+ferovost+pozitiv+komunikace AS Spirit,
	 nazev_mesta AS Město,
	 nazev_regionu AS Region
	FROM tym
	INNER JOIN statistika_tymu ON statistika_tymu.tymID = tym.tymID
	INNER JOIN mesto ON mesto.mestoID = tym.mestoID
	INNER JOIN region ON mesto.regionID = region.regionID
	GROUP BY nazev_regionu
	HAVING max(Spirit);
\end{lstlisting}

\subsubsection*{Které týmy hrají ve všech třech kategoriích?}

Každý tým může hrát v alespoň jedné nebo až třech kategoriích. Tento view slouží k ukázání týmů, které hrají ve všech třech kategoriích.

\begin{lstlisting}[language=SQL]
	CREATE VIEW tymyVsechKategorii AS
	SELECT
	 nazev_tymu AS Tým
	FROM tym
	INNER JOIN soutez ON soutez.tymID = tym.tymID
	INNER JOIN kategorie ON kategorie.kategorieID = soutez.kategorieID
	GROUP BY nazev_tymu
	HAVING count(nazev_tymu) = 3;
\end{lstlisting}

\pagebreak
\subsection*{Procedury}

Následují procedury pro zjednodušení různých úkonů v databázi. Rozdělené jsou na procedury pro hráče a pro týmy.

\subsubsection*{Procedury pro hráče}

První procedura vrátí produktivitu pro hráče daného jména. Druhá pak domovský tým tohoto hráče.

\begin{lstlisting}[language=SQL]
	CREATE PROCEDURE produktivitaHrace @jmeno TEXT, @prijmeni TEXT
	AS
	BEGIN
	 SELECT
	  jmeno AS jmeno,
	  prijmeni AS prijmeni,
	  (skore + asistence) / zapasu AS produktivita,
	 FROM hrac
	 INNER JOIN statistika_hrace ON statistika_hrace.hracID = hrac.rc
	 INNER JOIN tym ON hrac.tymID = tym.tymID
	 WHERE jmeno = @jmeno AND prijmeni = @prijmeni;
	END
	
	EXEC produktivitaHrace @jmeno = Michal, @prijmeni = David	
\end{lstlisting}

\begin{lstlisting}[language=SQL]
	CREATE PROCEDURE tymHrace @jmeno TEXT, @prijmeni TEXT
	AS
	BEGIN
	 SELECT jmeno, prijmeni, tym
	 FROM hrac
	 INNER JOIN tym ON hrac.tymID = tym.tymID
	 WHERE jmeno = @jmeno AND prijmeni = @prijmeni;
	END
	
	EXEC tymHrace @jmeno = Michal, @prijmeni = David
\end{lstlisting}

Poslední procedura přidá nového hráče do tabulky.

\begin{lstlisting}[language=SQL]
	CREATE PROCEDURE pridejHrace @jmeno TEXT, @prijmeni TEXT, @rc INTEGER, @tym TEXT
	AS
	BEGIN
	 SELECT tymID FROM tym WHERE nazev_tymu = @tym INTO @newID
	 INSERT INTO hrac (jmeno, prijmeni, rc, tymID)
	 	VALUES (@jmeno, @prijmeni, @rc, @newID)
	 COMMIT;
	END
	
	EXEC pridejHrace 'Petr', 'Partl', 9606138714, 'Prague Devils'
\end{lstlisting}

\pagebreak
\subsubsection*{Procedury pro týmy}

Následuje procedura, která pro vybraný tým spočítá jeho průměrný spirit.

\begin{lstlisting}[language=SQL]
	CREATE PROCEDURE spiritTymu @nazev_tymu TEXT
	AS
	BEGIN
	 SELECT
	  nazev_tymu AS jmeno,
	  pravidla+fauly+ferovost+pozitiv+komunikace AS spirit,
	 FROM tym
	 INNER JOIN statistika_tymu ON statistika_tymu.tymID = tym.tymID
	 WHERE jmeno = @nazev_tymu;
	END
	
	EXEC spiritTymu 'Chupacabras'
\end{lstlisting}

Poslední procedura je přidání daného týmu.

\begin{lstlisting}[language=SQL]
	CREATE PROCEDURE pridejTym @nazev_tymu TEXT, @nazev_mesta TEXT
	AS
	BEGIN
	 IF EXISTS(SELECT * FROM tym) THEN
	  SELECT MAX(tymID)+1 FROM tym INTO @newTymID;
	 ELSE
	  @newTymID=1;
	 END
	 SELECT mestoID FROM mesto WHERE nazev_mesta = @nazev_mesta INTO @newMestoID
	 INSERT INTO tym (tymID, nazev_tymu, mestoID)
	 	VALUES (@newTymID, @nazev_tymu, @newMestoID)
	 COMMIT;
	END
	
	EXEC pridejTym 'Flying Tigers', 'Brno'
\end{lstlisting}

\end{document}